\documentclass[a4paper,10pt]{article}
\usepackage[utf8]{inputenc}

%opening
\title{\textbf{Search for Standard Polarised Stars using RoboPol survey data}}

\author{\small{\textit{Siddharth Maharana(SM), IUCAA}} \\
\date{\small{\textsc{30 January, 2017}}}
}
\begin{document}

\maketitle{ }

\begin{abstract}
This document contains details of the analysis performed by me in order to find {Standard polarised star\labe(pol)} candidates from 3 years of RoboPol survey data. The objects in the fields 
of RoboPol images are afflicted with various systenatic effects, which are addressed in the codes. As a rough 
estimate, I expect to get about 70-100 such promising candidates, which will then be followed up by observations through RoboPol as the central object to ascertain their polarisation 
non-variability .    
\end{abstract}

\section{Introduction}

RoboPol has been taking data since late 2013 onwards. Although designed for observations of blazars and other point sources, owing to it's large field of view 
of $13\times13 arcminutes$, it has observed multiple other stars in each exposure. My analysis is aimed to look for candidate \ref{pol}

\end{document}
